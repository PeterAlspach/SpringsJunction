\documentclass[12pt]{article} % use larger type; default would be 10pt
\usepackage[utf8]{inputenc} % set input encoding (not needed with XeLaTeX)

%%% PAGE DIMENSIONS
\usepackage{geometry} % to change the page dimensions
\geometry{a4paper} % or letterpaper (US) or a5paper or....
\geometry{margin=2cm} % or letterpaper (US) or a5paper or....

\usepackage{graphicx} % support the \includegraphics command and options
\usepackage[parfill]{parskip} % Activate to begin paragraphs with an empty line rather than an indent
\usepackage{times} % for Times Roman default font

%%% PACKAGES
\usepackage{booktabs} % for much better looking tables
\usepackage{array} % for better arrays (eg matrices) in maths
\usepackage{paralist} % very flexible & customisable lists (eg. enumerate/itemize, etc.)
\usepackage{verbatim} % adds environment for commenting out blocks of text & for better verbatim
\usepackage{subfig} % make it possible to include more than one captioned figure/table in a single float

%%% HEADERS & FOOTERS
\usepackage{fancyhdr} % This should be set AFTER setting up the page geometry
\pagestyle{fancy} % options: empty , plain , fancy
\renewcommand{\headrulewidth}{0pt} % customise the layout...
\lhead{}\chead{}\rhead{}
\lfoot{}\cfoot{\thepage}\rfoot{}

\makeatletter
\renewcommand{\maketitle}{%
  \begin{center}
    {\bfseries{\scshape{\Large{\@title\par}}}}
  \end{center}
  \medskip
  \begin{flushright}
    {\@date\par}
  \end{flushright}
    \bigskip\hrule\vspace*{2pc}%
}
\makeatother

\hyphenation{Kiwi-bank} % otherwise it may get hyphenated as Ki-wibank

%%% END Article customizations

%%% The "real" document content comes below...

\title{Using the KiwiBog}
\date{\today} % Activate to display a given date or no date (if empty),
         % otherwise the current date is printed 

\begin{document}
\maketitle

\bigskip

\textbf{The KiwiBog is a strictly sit-down affair, with no exceptions} (men may pee outside).  Poos (and toilet paper) go down the larger hole at the back and pees the smaller hole at the front.  This will happen naturally if you sit comfortably back on the seat.

\bigskip

\begin{itemize}
 \item Cover faeces with a handful of wood shavings, using about a 1:1 volume ratio (wood shavings:faeces)
 \item After peeing, pour about half a cup of water down the pee hole (i.e., the small one).  There is a 2l bottle of water on the floor
 \item Should you soil the black plastic moulding, please wipe it clean with a bit of toilet paper
 \item Leave the toilet seat down when finished
 \item Leaving the toilet window open and the vent fan going ensures no offensive odours inside
 \item Please close the toilet door when you leave
 \item Empty the pee bucket (which is outside) if the house is going to be vacant
 \item The poo bucket only needs emptying when (almost) full
\end{itemize}

\bigskip

\textit{Nga mihi}

\end{document}
