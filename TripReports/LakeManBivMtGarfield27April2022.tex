\documentclass[12pt]{article} % use larger type; default would be 10pt
\usepackage[utf8]{inputenc} % set input encoding (not needed with XeLaTeX)

%%% PAGE DIMENSIONS
\usepackage{geometry} % to change the page dimensions
\geometry{a4paper} % or letterpaper (US) or a5paper or....
\geometry{margin=2cm} % or letterpaper (US) or a5paper or....

\usepackage{graphicx} % support the \includegraphics command and options
\usepackage[parfill]{parskip} % Activate to begin paragraphs with an empty line rather than an indent
\usepackage{times} % for Times Roman default font

%%% PACKAGES
\usepackage{booktabs} % for much better looking tables
\usepackage{array} % for better arrays (eg matrices) in maths
\usepackage{paralist} % very flexible & customisable lists (eg. enumerate/itemize, etc.)
\usepackage{verbatim} % adds environment for commenting out blocks of text & for better verbatim
\usepackage{subfig} % make it possible to include more than one captioned figure/table in a single float

%%% HEADERS & FOOTERS
\usepackage{fancyhdr} % This should be set AFTER setting up the page geometry
\pagestyle{fancy} % options: empty , plain , fancy
\renewcommand{\headrulewidth}{0pt} % customise the layout...
\lhead{}\chead{}\rhead{}
\lfoot{}\cfoot{\thepage}\rfoot{}

\makeatletter
\renewcommand{\maketitle}{%
  {\bfseries{\scshape{\Large{\@title\par}}}}
}
\makeatother

\hyphenation{Kiwi-bank} % otherwise it may get hyphenated as Ki-wibank

%%% END Article customizations

%%% The "real" document content comes below...

\title{Lake Man Biv and out via Mt Garfield: 26-28 April 2022}

\begin{document}
  \maketitle

We met John at Windy Point shortly after 09:30.  Leaving his car there, we returned to the crossing point at the rest area just north of the Engineers' Camp.  John crossed at a point that looked too deep for us, so we went upstream a bit (to more or less opposite where the sign on the Tui Track is) and found a good crossing.  By the time we had crossed and put our boots on (we crossed in Crocks) it was about 10:15.

We lunched a little beyond the Doubtful Hut to be well away from the resident sandflies there.  We were entertained by a `friendly' miromiro (tomtit).  The remaining trip to the Biv was as always - a bit tiring towards the end!  However, we arrived in time to make a good trench on the uphill side of the biv, put up a clothes line on the west wall, replenish the rodent bait, and do some `spring' cleaning.  There were about three pages of entries in the hut book (not all representing overnight stays) since our previous visit about 18 months ago.  So usage seems to have increased.

The biv was noticeably warmer inside than out and quite comfortable for 3 people (although not as good as Lucretia Biv).

Time to the biv was about 5$\frac{3}{4}$ hours, which included a good long break (about an hour) for lunch.

The following day we got away somewhat after 09:00.  We arrived at the saddle after about an hour and began the climb to Pt 1693.  Although this was a bit of a slog, route finding was very easy as the gradient was moderate, most of the long tussock could be avoided, and there was not much speargrass. Following a short breather and snack, we continued along to Mt Murray for lunch.  After lunch we dropped down to the saddle and over Pt 1726 (the highest elevation of the trip).  Camp at the tarns (Pt 1556) was reached shortly after 16:00, so we'd had about 6 hours walking that day.  Dinner was early and we were all in bed before 18:00, escaping the cold (although it was 5$^o$C, it felt much colder).

Thursday dawned fine and still, and we had a lovely breakfast in the early morning sun while the tent flies dried on the surrounding tussock.  The climb to Pt 1666, which we began around 09:00, was gentle and soon over.  Then it was onto Mt Garfield (1676) and Pts 1536, 1514, 1452, 1304 and 1239 where there was a clearing and a small tarn (along with a small rainwater collection set-up belonging to the Boyle Village Outdoor Education Centre).  Another idyllic lunch spot. From here there was a clear track almost all the way back to the main track along the Hope River.  Just before we emerged (at the true right of an unamed stream - the first one would cross if coming from Windy Point) the track disappeared.  This is probably deliberated as the bush there was very open, and it would not be good to have the Mt Garfield track clearly visible from that along the Hope River.

We arrived back the car shortly before 16:00, so another day of about 6 hours walking.

\begin{figure}[ht]
%\centering
\begin{minipage}{.5\linewidth}
\begin{flushleft}
   \includegraphics[width=8cm]{LakeManBivMtGarfield27April2022Photo1}
   \captionof{figure}{Tarn on the saddle before Pt 1693}
\end{flushleft}
\end{minipage}
\begin{minipage}{.5\linewidth}
\begin{flushright}
   \includegraphics[width=8cm, angle=270]{LakeManBivMtGarfield27April2022Photo2}
   \captionof{figure}{Glorious ridge walking}
\end{flushright}
\end{minipage}
\end{figure}

\begin{figure}[ht]
	%\centering
	\begin{minipage}{.5\linewidth}
		\begin{flushleft}
			\includegraphics[width=8cm]{LakeManBivMtGarfield27April2022Photo3}
			\captionof{figure}{View down to Doubtful Valley}
		\end{flushleft}
	\end{minipage}
	\begin{minipage}{.5\linewidth}
		\begin{flushright}
			\includegraphics[width=8cm]{LakeManBivMtGarfield27April2022Photo4}
			\captionof{figure}{More ridge walking}
		\end{flushright}
	\end{minipage}
\end{figure}

\begin{figure}[ht]
%\centering
\begin{minipage}{.5\linewidth}
\begin{flushleft}
   \includegraphics[width=8cm]{LakeManBivMtGarfield27April2022Photo5}
   \captionof{figure}{Morning campsite}
\end{flushleft}
\end{minipage}
\begin{minipage}{.5\linewidth}
\begin{flushright}
   \includegraphics[width=8cm]{LakeManBivMtGarfield27April2022Photo6}
   \captionof{figure}{Final lunch spot}
\end{flushright}
\end{minipage}
\end{figure}

\begin{flushright}
John, Robyn and Peter
\end{flushright}

\end{document}
